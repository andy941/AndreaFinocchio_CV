\documentclass{moderncv}
\usepackage[T1]{fontenc}
\usepackage[utf8]{inputenc}

\moderncvstyle{casual}
\moderncvcolor{blue}

\usepackage[scale=0.8]{geometry}
\recomputelengths

\firstname{Andrea}
\familyname{Finocchio}
\title{Research Scientist}
\address{11 Clanmahon road}{Killester, Dublin 05}
\mobile{+39 389 198 2299}
\email{finoccha@tcd.ie}
\extrainfo{\weblink{Linkedin: https://www.linkedin.com/in/andrea-finocchio-769a501b5/}}

\begin{document}
\maketitle
\section{Education}
\cventry{2018--2022}{PhD in Plant Genetics}{Trinity College Dublin}{Dublin}{}{}
\cventry{2016--2018}{Molecular Biotechnology and Bioinformatics}{Universitá degli Studi di Milano}{Milan}{\textit{110/110 with honors}}{}
\subsection{Master thesis}
\cvline{Title}{\emph{Role of ALOG gene family members during inflorescence development in rice}}
\cvline{Supervisor}{Prof.~Martin Kater, Department of Biosciences UNIMI}
\cventry{2013--2016}{Plants, Food and Agroenvironmental Biotechnology}{Universitá degli Studi di Milano}{Milan}{}{}
\subsection{Bachelor thesis}
\cvline{Title}{\emph{Identification and characterization of factors involved in the formation of protein bodies: the role of ATCYP21-2, a cyclofillin involved in the secretory pathway of Arabidopsis Thaliana}}
\cvline{Supervisors}{Emanuela Pedrazzini PhD (IBBA CNR), Prof. Alessio Scarafoni (Agraria UNIMI)}
%
\section{Experience}
\cventry{2020}{Data Analyst}{Enfer Group}{Naas}{}{Annotation and organization of COVID-19 test results for the major test provider in Ireland during the COVID-19 pandemic}
%
\section{Languages}
\cvlanguage{Italian}{Mother tongue}{}
\cvlanguage{English}{Highly proficient}{I have lived and worked in Dublin for four years}
\cvlanguage{Spanish}{Basic}{}
%
\section{Computer skills}
\cvcomputer{Data Science}{\texttt{Python, R, SQL}}{DevOps}{\texttt{Bash}}
\cvcomputer{Compiled}{\texttt{C++, C}}{Other}{\texttt{Git, CMake, \LaTeX{}}}
%
% Publications from a BibTeX file
\nocite{*}
\bibliographystyle{plain}
\bibliography{publications} % ’publications’ is the name of a BibTeX file
%
\end{document}
