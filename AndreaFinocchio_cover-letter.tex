\documentclass{moderncv}
\usepackage[T1]{fontenc}
\usepackage[utf8]{inputenc}

\moderncvstyle{casual}
\moderncvcolor{blue}

\usepackage[scale=0.7]{geometry}
\recomputelengths

\firstname{Andrea}
\familyname{Finocchio}
\title{Research Scientist}
\address{11 Clanmahon road}{Killester, Dublin 05}
\mobile{+39 389 198 2299}
\email{finoccha@tcd.ie}
\extrainfo{\weblink{Linkedin: https://www.linkedin.com/in/andrea-finocchio-769a501b5/}}

\begin{document}
\maketitle
\section{About me}
I have been a scientist my all life. After a bachelor in plant biotechnology
and a master in bioinformatics at the University of Milan, I moved to Dublin
and applied my skills to generate and analyse complex transcriptomics and
proteomics datasets, from which I am planning to graduate at the end of the
year.

My interests lay where biology meets programming. I have strong programming
skills in C++, R, Python, Bash, together with the knowhow to produce genomics,
transcriptomics and proteomics datasets. During my PhD, I leveraged cutting
edge third generation sequencing technologies to study complex alternative
splicing events during the first stages of flower development, and used and
biotin labeling coupled with mass spectrometry to explore the interactomes
around major players in trichome development. In the four years of my PhD I
taught myself all the programming skills needed to solve the problems I faced.
I familiarized with GPU and cluster computing, learnt R, Python and Bash to
analyze the output of the sequencing and mass spectrometry experiments and set
up the Git collaboration pipeline that coordinates my lab efforts to analyze
our -omics datasets.

During the COVID-19 outbreak, I was hired at Enfer group, the major covid
testing hub in Ireland, and was responsible for the data analysis department.
During my time there, I annotated and organized hundreds of thousands of test
results in an SQL database, improving traceability and data retrieval speed.

Learning is the thing I strive for in life. In my spare time I like to learn
new skills and continue to improve in areas that fascinate me. I completed
certified courses in machine learning, statistics and Bayesian frameworks.
More than a year ago I expanded my programming skills by learning C++, which I
am currently applying to implement the most common bioinformatics algorithms.
A record of my journey can be found on my GitHub page
(\weblink{https://github.com/andy941}).

\end{document}

