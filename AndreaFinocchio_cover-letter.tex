\documentclass{moderncv}
\usepackage[T1]{fontenc}
\usepackage[utf8]{inputenc}

\moderncvstyle{casual}
\moderncvcolor{blue}

\usepackage[scale=0.8]{geometry}
\recomputelengths

\firstname{Andrea}
\familyname{Finocchio}
\title{Research Scientist}
\address{11 Clanmahon road}{Killester, Dublin 05}
\mobile{+39 189 198 2299}
\email{finoccha@tcd.ie}
\extrainfo{\weblink{https://github.com/andy941}}

\begin{document}
\maketitle
\section{About me}
Research scientist, currently undertaking a PhD in plant genetics. After a
bachelor in plant biotechnology and a master in bioinformatics, I moved to
Dublin and applied my skills to produce and analyse complex transcriptomics
and proteomics datasets. My interests lay where biology meets programming. I
have strong programming skills in C++, R, python, bash, together with the know
how to produce genomics, transcriptomics and proteomics datasets. My PhD work
revolves around the study of changes in the transcriptome in the first stages
of flower development using third generation sequencing technologies and a
biotin labeling experiment to explore the interactome around major players in
trichome development. Learning is the thing I strive for in life. In the 4
years of my PhD I taught myself all the programming skills needed to solve the
problems I faced, including to familiarize with GPU and cluster computing. I
had set up the Git collaboration pipeline that coordinates my lab efforts to
analyze our -omics datasets for a more robust and agile version control
system. Covid hit right in the middle of my PhD and disrupted my work
completely for 3 months. During that time, I was hired at Enfer group, the
major covid testing hub in Ireland and being responsible of the data analysis
department. During my time there, I annotated and organized hundreds of
thousands of test results in an SQL database, improving traceability and data
retrieval performance. In my spare time I like to learn new skills and improve
in areas that fascinate me. In the years of my PhD, I took certified courses
in machine learning and Bayesian statistics and I greatly expanded my
programming skills by learning C++, which I am currently applying to implement
the most common bioinformatics algorithms. A record of my journey in
programming can be found on my GitHub page.
\end{document}

